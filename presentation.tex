
\documentclass{beamer}
\usepackage{multicol}
\usepackage{braket}
\usepackage{tikzit}
\input{zx-calculus.tikzstyles}
\usepackage{tikz}


\usetheme{metropolis}           % Use metropolis theme
\title{Reconfiguring staggerd quantum walks with ZX}
\date{November 5, 2024}
\author{Bruno Jardim \and Jaime Santos \and Luís Soares Barbosa}
\institute{HASLab - INESCTEC}
\begin{document}
\maketitle
\section{Introduction to the Staggerd Quantum Walk}
\begin{frame}{Staggerd Quantum Walk}
	In contrast to conventional, coin-based quantum walks, which proceed straightforwardly from one vertex to another, the staggered variant takes advantage of forming partitions of graph cliques over the graph structure of the walking space.
\end{frame}


\section{Introduction to the ZX-Calculus}
\begin{frame}{ZX-Calculus}
	The ZX-calculus is diagrammatic language for reasoning about linear maps between qubits and, as such, about quantum computation in general.


\end{frame}

\begin{frame}{ZX-Calculus - Generators}
	\tikzfig{zx-spiders}
\end{frame}
\begin{frame}{ZX-Calculus - Rewrite Rules}



	\begin{multicols}{2}
		\begin{itemize}
			\item Spider Fusion
			      \resizebox{0.4\textwidth}{!}{% 
				      \tikzfig{rule-sf}
			      }%
			\item Identity Removal
			      \resizebox{0.4\textwidth}{!}{% 
				      \tikzfig{rule-id}
			      }%
			\item Color Change
			      \resizebox{0.4\textwidth}{!}{% 
				      \tikzfig{rule-colour-change}
			      }%
			\item Hadamard Identity
			      \resizebox{0.4\textwidth}{!}{% 
				      \tikzfig{rule-had-id}
			      }%
			\item Bialgebra
			      \resizebox{0.4\textwidth}{!}{% 
				      \tikzfig{rule-bialg}
			      }%
			\item $\pi$-commutation
			      \resizebox{0.4\textwidth}{!}{% 
				      \tikzfig{rule-pi-com}
			      }%
			\item Hopf
			      \resizebox{0.4\textwidth}{!}{% 
				      \tikzfig{rule-hopf}
			      }%
			\item State copy
			      \resizebox{0.4\textwidth}{!}{% 
				      \tikzfig{rule-state-copy}
			      }%


		\end{itemize}
	\end{multicols}



\end{frame}
\section{Bringing ZX into the picture}

\begin{frame}{Staggerd Quantum Walk - Circuit}
	A general implementation of a Staggerd Quantum Walk for a line graph.

	\ctikzfig{sqw-circuit}

\end{frame}


\begin{frame}{Staggerd Quantum Walk - ZX-diagram}
	A concrete implementation of a Staggerd Quantum Walk for a line graph with 3 qubits and the state $\ket{4}$ as the initial state.

	\ctikzfig{1-step-sqw}

\end{frame}
\begin{frame}{Staggerd Quantum Walk - ZX-diagram}
	The previous diagram utilizes a notation from the ZH-calculus for the Tofolli gates that greatly simplifies the diagram. Expanding the Tofolli gates it yields
	\begin{align*}
		\resizebox{0.9\textwidth}{!}{% 
			\tikzfig{1-step-expanded}
		}%
	\end{align*}
\end{frame}
\begin{frame}{Staggerd Quantum Walk - ZX-diagram}
	\begin{align*}
		\resizebox{0.9\textwidth}{!}{% 
			\tikzfig{1-step-trivial}
		}%
	\end{align*}
\end{frame}

\end{document}
